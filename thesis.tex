% Opcje klasy 'iithesis' opisane sa w komentarzach w pliku klasy. Za ich pomoca
% ustawia sie przede wszystkim jezyk i rodzaj (lic/inz/mgr) pracy, oraz czy na
% drugiej stronie pracy ma byc skladany wzor oswiadczenia o autorskim wykonaniu.
\documentclass[declaration,shortabstract,mgr,masc,english]{iithesis}

\usepackage[utf8]{inputenc}

%%%%% DANE DO STRONY TYTUŁOWEJ
% Niezaleznie od jezyka pracy wybranego w opcjach klasy, tytul i streszczenie
% pracy nalezy podac zarowno w jezyku polskim, jak i angielskim.
% Pamietaj o madrym (zgodnym z logicznym rozbiorem zdania oraz estetyka) recznym
% zlamaniu wierszy w temacie pracy, zwlaszcza tego w jezyku pracy. Uzyj do tego
% polecenia \fmlinebreak.
\polishtitle    {Parsing zależnościowy za pomocą sieci neuronowej}
\englishtitle   {Neural dependency parsing}
\polishabstract {Prezentujemy parser zależnościowy zaimplementowany jako pojadyńcza
sieć neuronowa, która działa bezpośrednio na literach słów. W przeciwieństwie
do typowych sposobów parsingu nasz model nie wymaga żadnego pre-processingu danych,
(w szczególności tagowania słów). Dodatkowo prezentujemy modele wytrenowane na
dwóch językach jednocześnie, które uzyskują wyniki konkurencyjne do najlepszych
dostępnych parserów.}
\englishabstract{We present a dependency parser, implemented as single neural
network which works directly on character representation of the words. Unlike
typical approches our model doesn't require any preprocessing of the data, like
part-of-speech tagging. Additionally we present models trained jointly on two
languages, which can obtain state-of-the art results.}
% w pracach wielu autorow nazwiska mozna oddzielic poleceniem \and
\author         {Michał Zapotoczny}
% w przypadku kilku promotorow, lub koniecznosci podania ich afiliacji, linie
% w ponizszym poleceniu mozna zlamac poleceniem \fmlinebreak
\advisor        {dr Jan Chorowski}
%\date          {}                     % Data zlozenia pracy
% Dane do oswiadczenia o autorskim wykonaniu
\transcriptnum {248100}                     % Numer indeksu
\advisorgen    {dr. Jana Chorowskiego} % Nazwisko promotora w dopelniaczu
%%%%%

%%%%% WLASNE DODATKOWE PAKIETY
%
% \usepackage{import,graphicx,listings,amsmath,amssymb,amsthm,amsfonts,tikz}
\usepackage{graphicx,import,microtype}%,import,color,microtype}
\usepackage{amsmath,amsthm,floatrow}
\let\lll\undefined
\usepackage{amssymb} % http://www.rexamine.com/2013/04/latex-polish-babel-and-amssymb-conflict/
\usepackage{wrapfig}
\usepackage{todonotes}
\usepackage{multirow}
\usepackage{transparent}
\usepackage[boxed]{algorithm2e}
%
%%%%% WŁASNE DEFINICJE I POLECENIA
%
%\theoremstyle{definition} \newtheorem{definition}{Definition}[chapter]
%\theoremstyle{remark} \newtheorem{remark}[definition]{Observation}
%\theoremstyle{plain} \newtheorem{theorem}[definition]{Theorem}
%\theoremstyle{plain} \newtheorem{lemma}[definition]{Lemma}
%\renewcommand \qedsymbol {\ensuremath{\square}}
% ...
\newcommand{\todotext}[1]{{\color{red}#1}}
\newcommand{\txtrightarrow}{$\,\to\,$}
\SetKwProg{Funct}{Function}{}{}
%%%%%
\begin{document}

%%%%% POCZĄTEK ZASADNICZEGO TEKSTU PRACY
\chapter{Introduction}
The ability to communicate with the user in a natural language is a major driving
force in development of natural language processing algorithms. One of the basic
tasks in a NLP pipeline is parsing by which we can describe structure of sentences.
There exist two basic parsing techniques: constituency and dependency parsing.
With a constituency parser we break the sentence into phrases, which can be further
broken into smaller sub-phrases. Example of such parsing is shown in Figure~\ref{fig:constituency_tree}.

\begin{figure}[!htbp]
  \centering
  \resizebox{0.5\textwidth}{!}{
    \import{img/examples/dep/}{constituent.pdf_tex}
  }
  \caption{A sample constituency parse tree} 
  \label{fig:constituency_tree}
\end{figure}

In dependency parsing each word (called the \emph{dependent})
is connected via a labelled arc to another word of the sentence (called the \emph{head})
or to the special \emph{ROOT} vertex, forming a directed tree.
Example of such parsing is depicted in Figure~\ref{fig:dependency_tree}.

\begin{figure}[!htbp]
  \centering
  \resizebox{0.8\textwidth}{!}{
    \import{img/examples/dep/}{dep.pdf_tex}
  }
  \caption{A sample dependency parse tree} 
  \label{fig:dependency_tree}
\end{figure}

\section{Dependency parsing}

A major advantage of dependency parsers over constituency ones are their
independence from the word ordering. It is important in morphologically rich
languages like Polish or Czech where the ordering is very flexible, and thus
related words can be far apart from each other.
Additionally the head-dependent relationship is a good approximation to semantic
relationship between words~\cite{covington_fundamental_2001} which is important
for tasks like question answering or information extraction.

The labels of head-dependent arcs tells us about grammatical function that
dependent word have in respect to the head. In table~\ref{tab:label_samples}
we show some of the most popular labels for the English language.

\begin{table}[!htbp]
    \centering
    \begin{tabular}{c | c | c}
        Label & Description & Example \\ \hline\hline
        CASE & Case marking & \textit{From} Friday 's Daily \textbf{Star} \\
        NSUBJ & Nominal subject & \textit{Musharraf} \textbf{calls} the bluff \\
        NMOD & Nominal modifier & India \textbf{defensive} over Sri \textit{Lanka} \\
        DET & Determiner & That he missed \textit{a} \textbf{physical} ? \\
        DOBJ & Direct object & Did you \textbf{know} \textit{that} ? \\
        ADVMOD & Adverbial modifier & \textit{So} Bush \textbf{stopped} flying . \\
        AMOD & Adjectival modifier & Six weeks of \textit{basic} \textbf{training} . \\
        COMPOUND & Compound & Bush 's \textit{National} \textbf{Guard} years
    \end{tabular}
    \caption{Most popular labels from English treebank. Dependents are \textit{italic},
    while heads are \textbf{bold}}
    \label{tab:label_samples}
\end{table}

\subsubsection{Projectivity}
We can distinguish two types of dependency trees: projective and nonprojective.
We say that head-dependent arc is projective is there exists a path from head
to every word between head and dependent in the sentence. A dependency tree is
projective if all its arcs are projective.
Intuitively we can say that nonprojective trees are trees that cannot be drawn
without crossing the edges (see Figure~\ref{fig:dependency_nonproj}).
Although English dataset has less than 5\% of nonprojective sentences other
languages can have significant amount - like Czech (12\%) or Ancient Greek (63\%)~\cite{straka_parsing_2015}.
It is important to acknowledge this fact because some of the presented parsing
methods can only produce projective trees.

\begin{figure}[!htbp]
  \centering
  \resizebox{1.0\textwidth}{!}{
    \import{img/examples/dep/}{nonproj.pdf_tex}
  }
  \caption{A sample nonprojective dependency parse tree. The on $\rightarrow$ hearing arc is
    nonprojective and thus the whole tree becomes nonprojective.}
  \label{fig:dependency_nonproj}
\end{figure}

\subsection{Parsing algorithms}
In the following section we will present two basic approaches to dependency parsing.
Although they differ in complexity and output flexibility, both of them depend
on supervised machine learning techniques.

\subsubsection{Transition based}
A configuration $c$ is a triple $(s, b, A)$ containing stack $s$,
buffer $b$ and set of dependency arcs $A$.
In transition based parsing we find a transition sequence from an initial configuration
to a terminal one.  For given sentence $w_1, \cdots w_n$
the initial configuration is $(\text{[ROOT]}, [w_1, \cdots w_n], \emptyset)$.
and the terminal one is $(\text{ROOT}, [], A)$ in which $A$ is a resulting parse tree.
There are three possible transitions:
\begin{itemize}
    \item {\ttfamily LEFTARC(l)} - Pop from the stack elements $s_1$ and $s_2$,
        add arc $s_1 \rightarrow s_2$ with label $l$ to the set $A$ and push $s_1$ back
        to the stack
    \item {\ttfamily RIGHTARC(l)} - Pop from the stack elements $s_1$ and $s_2$,
        add arc $s_2 \rightarrow s_1$ with label $l$ to the set $A$ and push $s_2$ back
        to the stack
    \item {\ttfamily SHIFT} - Move first element from the buffer to the stack
\end{itemize}
An example of parsing sequence is shown in table~\ref{tab:transition_parse}.
\begin{table}[!htbp]
    \centering
{\footnotesize
    \begin{tabular}{l | l | l | l}
        Stack & Buffer & Transition & New arcs \\ \hline
        $[$ROOT$]$ & $[$He has good control$]$ & {\ttfamily SHIFT} & - \\
        $[$ROOT He$]$ & $[$has good control$]$ & {\ttfamily SHIFT} & - \\
        $[$ROOT He has$]$ & $[$good control$]$ & {\ttfamily LEFTARC(nsubj)} & nsubj(has,He) \\
        $[$ROOT has$]$ & $[$good control$]$ & {\ttfamily SHIFT} & - \\
        $[$ROOT has good$]$ & $[$control$]$ & {\ttfamily SHIFT} & - \\
        $[$ROOT has good control$]$ & $[$$]$ & {\ttfamily LEFTARC(amod)} & amod(control, good) \\
        $[$ROOT has control$]$ & $[$$]$ & {\ttfamily RIGHTARC(dobj)} & dobj(has, control) \\
        $[$ROOT has$]$ & $[$$]$ & {\ttfamily RIGHTARC(root)} & root(ROOT, has) \\
        $[$ROOT$]$ & $[$$]$ & - & - \\
    \end{tabular}
}
    \caption{Transitions needed to parse sentence "He has good control"}
    \label{tab:transition_parse}
\end{table}

Because every word of the sentence has to be shifted to the stack exactly once
and {\ttfamily ARC} operations reduce stack size by one we will have exactly
$2n$ transitions from initial to terminal configuration. Thus the time complexity of this
algorithm is $O(n)$ where $n$ is number of words in a sentence.
Using this algorithm we can represent any projective tree 
\cite{nivre_algorithms_2008} (nonprojectives cannot be represented).

\begin{sloppypar}
The decision which transition to use between consecutive configurations is
obtained through machine learning techniques. This can vary from simple
linear SVM~\cite{nivre_maltparser:_2005} to neural networks~\cite{chen_fast_2014,
andor_globally_2016}.
\end{sloppypar}

\subsubsection{Graph based}
Alternative approach is to use graph-based algorithms. The idea is that we define
a space of candidate arcs, find a model that scores each arc and then use some tree building 
algorithm to find a final dependency tree with the highest score.

The candidate space and scoring depends on the used learning model.
There exists many parsing algorithms, but one of the most popular is
Chu-Liu-Edmonds \cite{edmonds_optimim_1966}.
The Chu-Liu-Edmonds algorithm, given a directed graph produces minimum spanning
arborescence (minimum spanning tree for directed graph) - it allows us to produce
any parse tree (including nonprojective ones). 

Formally we present this method as Algorithm \ref{lst:cle}, but the general 
idea is that each vertex chooses 
an incoming edge of minimum weight. If this graph does not contain any cycle then
we are done, else we break the cycle in place that would give us the best result.


\begin{algorithm}[!htbp]
    \KwData{Directed graph $G = (V,E)$, root node $r \in V$, weight function $w(e), e \in E$}
    \KwResult{Spanning arborescence of minimum weight rooted at $r$}
    \Funct{edmonds($G=(V,E)$, $r$, $w$)}{
        Remove any edge from $G$ whose destination is $r$ \;
        For each $v \in V - \{r\}$ find incoming edge of minimum weight, denote it as $\pi(v)$ \;
        Denote new set of edges as $P = {(\pi(v), v) | v \in V - \{r\})}$ \;
        \eIf{$P$ does not contain a cycle}{
            \Return $P$
        }{
            Choose any cycle from $P$ and denote it as $C=(C_v,C_e)$ \;
            Define a new directed graph $G'=(V',E')$, and weighting function $w'$, where $V' = V - C_v + {v_c}$ \;
            Define $E'$ and $w'$ as follows:
            \For{$(u,v) \in E$} {
                \uIf{$u \not\in C_e, v \in C_e$}{
                    $E' = E' + (u, v_c), w'(u, v_c) = w(u,v) - w(\pi(v), v)$
                }
                \uElseIf{$u \in C_e, v \not\in C_e$}{
                    $E' = E' + (v_c, v), w'(v_c, v) = w(u,v)$
                }
                \ElseIf{$u \not\in C_e, V \not\in C_e$}{
                    $E' = E' + (u, v), w'(u,v) = w(u,v)$
                }
                For each new edge remember its origin \;
            }
            Denote $A' = \text{edmonts}(G', r, w')$ \;
            Let $(u, v_c)$ be (unique) incoming edge to $v_c$ in $A'$ \;
            Let $(u,v) \in E (v \in C)$ be a corresponding edge from $G$ \;
            Remove edge $(\pi(v), v)$ from $C$ (breaking the cycle) \;
            Each remaining edge in $C$ together with edges from $E$ that are also present
            in $A'$ forms a minimum spanning arborescence. \;
        }
    }
 \caption{Chu-Liu-Edmonds Algorithm.The standard algorithm has $O(EV)$ running time, but there exist faster implementation
    by \cite{gabow_efficient_1986} with $O(E + V\log V)$ running time. }
 \label{lst:cle}
\end{algorithm}

\subsection{Evaluation of parsing algorithms}
Having predicted dependency tree for a particular sentence we have to be able
to compare it with the gold-standard parse.
There are two main evaluation metrics:
\begin{itemize}
    \item \textbf{Unlabelled Attachment Score} (UAS), is a percentage of words
        with correct predicted head
    \item \textbf{Labelled Attachment Score} (LAS), is a percentage of words
        with correct predicted head \textbf{and} correct predicted label
\end{itemize}
In the experimental section we will show both scores, whenever available.

\subsection{Universal Dependencies}
Up till recently most development of dependency parsers was done in single language
setting, where we have different parser for every language we want to use, each
using language-specific morphosyntactical features. The Universal Dependencies
project \cite{nivre_universal_2015} aims to provide a cross-linguistically consistent
treebank annotation. It is based on previous work on universal Stanford
Dependencies~\cite{marneffe_generating_2006},
Google universal pos tags~\cite{petrov_universal_2011}
and the Interset interlingua~\cite{zeman_reusable_2008}.
The UD project unifies part of speech tags and dependency relations labels
for 30+ languages under common CONNL-U format.
The main advantage is that we can use the same parsing system for every language
(data has consistent format) additionally because the morphosyntactical data has
common format we can try to combine data coming from different languages (see
section~\ref{sec:neural_multilingual}).

\section{Neural Networks}
A neural network is a mathematical model loosely based on how the brain works.
The models were proven to be very flexible and obtained state of the art results in
many tasks like machine translation~\cite{bahdanau_neural_2014},
caption generation~\cite{xu_show_2015} or image recognition~\cite{szegedy_goglenet_2014}
A neural network is defined by its topology,
activation functions and parameters. The first two are chosen by the author,
while the parameters are obtained by a learning procedure.

The topology of the network is telling us how the layers of neurons are connected
to each other. Each layer has a number of neurons which have some incoming
indices $i$, associated activation function $\phi$ and some parameters $w$.
The neuron combines the incoming signals, usually by taking weighted average
and then applying the activation function: $ \phi ( \sum_{k=0}^{|i|} w_ki_k ) $.
This value is then fed to the connected neurons from the next layer.

The activation function can have any form, but the most common are hyperbolic tangent
$\tanh$ and ReLU \cite{nair_rectified_2010}.

By the learning procedure we understand an optimization process that given a
loss function $L$, data $x$ and some parameters $W$ to tune, tries to minimize $L(x)$.
Because usually the data $x$ size is big we cannot use simple gradient descent
algorithm on all data at once because it would not fit into the memory. Instead
we are using stochastic gradient descent to train only on small subset of data
at once (this subset is called a minibatch).

\begin{figure}[!htbp]
  \centering
  \includegraphics[width=0.6\linewidth]{img/nn/neural_net.jpeg}
  \caption{An sample neural network architecture} 
  \label{fig:sample_nn}
\end{figure}


\chapter{Neural dependency parser}
In this chapter we will present the overall topology of our neural network.
Additionally we will describe the training procedure and how the proper parameters
were selected.

\section{Overview of the network architecture}
The network architecture consists of three main parts: reader, tagger and parser
(see Figure~\ref{fig:architecture}). The reader subnetwork is evaluated on
each individual word in a sentence, and using convolutions on their orthographic
representation produces words embeddings. Next, the tagger subnetwork implemented
as bidirectional RNN equips each word with a context of whole sentence. Finally
parser part computes dependency tree parent for each word using attention mechanism
\cite{vinyals_pointer_2015} after which network computes appropriate dependency label.
In the following paragraphs we will describe all parts in detail.

\begin{figure}[!htbp]
  \centering
  \resizebox{0.8\textwidth}{!}{
    \import{img/network/}{drawing.pdf_tex}
  }
  \caption{The model architecture.} 
  \label{fig:architecture}
\end{figure}

\subsection{Reader}
As stated before the reader subnetwork is run on each word producing its embedding.
This architecture is based on~\cite{kim_character-aware_2015}. Each word $w$
is represented by sequence of its characters plus a special beginning-of-word \texttt{<bow>} and
end-of-word \texttt{<eow>} tokens. First, we find low-dimensional characters embeddings which
are concatenated to form a matrix $C^w$. Then we run 1D convolutional filters
on $C^w$ which then is reduced to vector of filter responses computed as:
\begin{equation} \label{eq:filter_responses}
    R^w_i = \max( C^w \ast F^i )
\end{equation}
Where $F^i$ is i-th filter. The purpose of the convolutions is to react to
specific part of words (because we have \texttt{<bow>} and \texttt{<eow>} tokens
it can also react to prefixes and suffixes) which in morphologically rich languages such as Polish
can depict its grammatical role.

Finally we transform filter responses $R^w$ with a simple multi-layer perceptron
\footnote{Which are just linear transformations followed be non-linearity}
obtaining the final word embedding $E^w = \text{MLP}(R^w)$.

\subsection{Tagger}
\begin{sloppypar}
Having obtained the word embeddings $E^w$ we can proceed with actually "reading"
whole sentence. To do this we use multi-layer bidirectional RNN (\cite{schuster_bidirectional_1997})
(we have evaluated LSTM~\cite{RNN_LSTM} and GRU~\cite{RNN_GRU} types). We combine
the backward and forward passes by adding them.
\end{sloppypar}

\subsubsection{POS tag predictor}
To prevent overfitting and encourage network to compute morphological features
we can add additional training objective. It works by taking output from one of
the tagger BiRNN layers and use it to predict available part-of-speech tags for
each word. The result is not feeded back to the rest of the network because we
think it would introduce too much noise.

\subsection{Parser}
The last part of the network serves two purposes: to find head for each word and
to compute label for that edge.

\subsubsection{Finding the head word}
We use a method similar to~\cite{vinyals_pointer_2015}.
The input to this part are word annotations $H_0, H_1, \dots, H_n$ (where $H_0$
serves as a root word) produced by the tagger. For each of the words $1,2,\dots,n$
we compute probability distribution which tell us where the head of current
word should be ($0,1,2,\dots,n$). This computation (called \emph{scorer}) is implemented
as small feedforward network $s(w,l) = f(H_w, H_l)$, where $w \in {1,2,\dots,n}$,
$l \in {0,1,2,\dots,n}$.

\subsubsection{Finding edge label}
Output of the scorer can already be interpreted as
pointer network, but in order to use it as attention for computing dependency label
we have to normalize it:
\begin{equation} \label{eq:label_attention}
    p(w,l) = \sum_{i=0}^{n} \frac{f(H_w, H_l)}{f(H_w, H_i)}
\end{equation}
The dependency label is computed by small Maxout network~\cite{goodfellow_maxout_2013},
which takes the current word annotation $H_w$ and heads annotation $A_w$. This
part is called the \emph{labeller}.
We investigated two variations of head annotation $A_w$
\\
\\
\\
\emph{Soft attention}\\
Which is a weighted average of normalized attention~\ref{eq:label_attention}
and words annotations $H$. 
$$ A_w = \sum_{i=0}^{n} p(w,i)H_i $$
\\
\emph{Hard attention}\\
Here we only use a head annotation.
$$ A_w = H_h $$
During training we use ground-truth head location, whereas during evaluation
we use head word computed in previous step.

\subsubsection{Decoding algorithm}
The \emph{scorer} give us a $n$x$n+1$ matrix of head dependency preference
for each word. From this matrix we have find a set of dependencies that satisfy
some constraints (exactly one root dependant, no cycles).
We investigated two possibilities for such decoding: a greedy algorithm, and
Chu-Liu-Edmonds~\cite{edmonds_optimim_1966}.

\section{Training}
\subsection{Training criterion}
Every neural network needs a training criterion which will be optimized.
Here we have 3 individual training criterion combined together as linear
combination. Those are:
\begin{itemize}
        \item The negative log-likelihood loss $L_h$ on finding the proper head for each
            word. The training signal is propagated from scorer down to the reader.
        \item The negative log-likelihood loss $L_l$ on finding the dependency label.
            With a soft attention it is propagated through the whole network (excluding
            pos tagging part), while with the hard attention we do not propagate
            through the scorer.
        \item The (optional) negative log-likelihood loss $L_t$ on predicting pos tags.
            This error is backpropagated only through pos-predictor and part of tagger
            down to the reader.
\end{itemize}
The final loss is:
\begin{equation}\label{eq:neural_loss}
    L = \alpha_hL_h + \alpha_lL_l + \alpha_tL_t
\end{equation}

\subsection{Regularization, optimizer and hyperparameters} \label{sec:basic_params}
Regularization is an essential part of  neural network training. It improves 
generalization and prevents overfitting. One of the most popular regularization
technique is called Dropout~\cite{srivastava_dropout:_2014}. Using it, we randomly
drop part of the connections from a certain network layer during training.
In our case dropout is applied to the \emph{reader} output, between the BiRNN
layers of the \emph{tagger} and to the \emph{labeller}.

The models are trained using Adedelta~\cite{zeiler_adadelta:_2012} learning rule,
with weight decay and adaptive gradient clipping~\cite{chorowski_end--end_2014}.
All experiments are early stopped on validation set UAS.

Hyperparameter selection is crucial for neural networks to obtain good results.
For example, comparing our first experiments with architecture depicted above to
the best single-language results (using the same basic architecture) gave us
about 5\% boost in UAS score on Polish language.

\begin{sloppypar}
To find the best hyperparameters we have used the Spearmint system~\cite{snoek_practical_2012}
invoked on polish dataset. The chosen parameters are as follows.
The \emph{reader} embeds each character into 15 dimensions, and
contains 1050 filters (50$\cdot$k filters of length k for k = 1, 2,\dots, 6) 
whose outputs are projected into 512 dimensions transformed by a 3 equally
sized layers of feedforward neural network with ReLU activation.
The \emph{tagger} contains 2 BiRNN layers of GRU units with 548 hidden
states for both forward and backward passes which are later aggregated using
addition. Therefore the hidden states of the tagger are also 548-dimensional.
The \emph{POS tag predictor} consists of a single affine transformation
followed by a SoftMax predictor for each POS category.
The \emph{scorer} uses a single layer of 384 tanh for head word
scoring while the \emph{labeller} uses 256 Maxout units
(each using 2 pieces) to classify the relation label
\cite{goodfellow_maxout_2013}. The training cost used the constants
$\alpha_h=0.6, \alpha_l=0.4, \alpha_t=1.0$.
We apply 20\% dropout to the \emph{reader} output, 70\% between the BiRNN
layers of the \emph{tagger} and 50\% to the \emph{labeller}. The weight
decay is 0.95. %Adadelta epsilon annealed from 1e-8 to 1e-12
\end{sloppypar}

\section{Multilingual training} \label{sec:neural_multilingual}
The big problem of learning to parse is small number of gold standard dependency
trees for many languages (including Polish). With small number of examples
neural networks do not generalize well and can more easily overfit.
There also exist languages with good, standardized treebank like Czech Prague
Treebank~\cite{prague_treebank}.

Because our model is purely neural network we can incorporate a multitask learning
\cite{caruana_multitask_learning}. It allows the network to learn multiple tasks
at the same time, sharing common patterns and distinguishing differences.
Additionally, because we are using Universal Dependencies treebanks~\cite{nivre_universal_2015}
we have common standardized format across many languages, which allowed for
easy implementation of multitask learning and experiments across different languages.

The multilingual model for $n$ languages can be viewed as $n$ copies of our
basic model, but sharing part of the parameters. To unify input/output for each
of the models we sum possible outputs for each data category (characters,
POS categories, dependency labels). If some category doesn't exist within
a particular language, we use a special \texttt{UNK} token. 
To actually make use of multitask learning we must share at least part of the
parameters of all models. We experimented with different sharing strategies
, from share-everything to only sharing the  \textit{parser} part. Additionally
to prevent over-representation of some
languages during training we sample (on each epoch) only portion of the available
data so that each language have equal number of examples (equal to number of samples
of the smallest language).

\chapter{Results}

\section{Single language}
Our basic results for subset of UD v1.3\cite{nivre_universal_2015} as shown
in table \ref{tab:universal_basic}. We compare ourselves to SyntaxNet and
ParseySaurus, ....
\begin{table}[tb]
  \centering
  \caption{Baseline results of models trained on single languages from
    UD v1.3. Our models use only the orthographic representation of
    tokenized words during inference and can work without a separate POS tagger.}
  \label{tab:universal}
  \begin{tabular}{l l | l l | l l | l l}
    language & \#sentences & \multicolumn{2}{c|}{Ours} &
      \multicolumn{2}{c|}{SyntaxNet} & \multicolumn{2}{c}{ParseySaurus} \\ \hline
    & & UAS & LAS & UAS & LAS & UAS & LAS\\ \hline
    Czech & 87 913 & \textbf{91.41} & \textbf{88.18} & 89.47 & 85.93 & 89.09 & 84.99 \\
    Polish & 8 227 & 90.26 & 85.32 & 88.30 & 82.71 & \textbf{91.86} & \textbf{87.49}\\
    Russian & 5 030 & 83.29 & 79.22 & 81.75 & 77.71 & \textbf{84.27} & \textbf{80.65} \\
    German & 15 892 & 82.67 & 76.51 & 79.73 & 74.07 & \textbf{84.12} & \textbf{79.05}\\
    English & 16 622 & 87.44 & 83.94 & 84.79 & 80.38 & \textbf{87.86} & \textbf{84.45}\\ % Zblizamy sie do saurusa a dopiero pratraining...
    French & 16 448 & \textbf{87.25} & \textbf{83.50} & 84.68 & 81.05 & 86.61 & 83.1\\
    Ancient Greek & 25 251 & \textbf{78.96} & \textbf{72.36} & 68.98 & 62.07 & 73.85 & 68.1
  \end{tabular}
\end{table}
predictor, reader impact

\subsection{Soft vs Hard attention}
\subsection{Decoding algorithm}
\subsection{Word pieces}
\subsection{Pointer softening}
\subsection{Recurrent state size}

\section{Multilanguage}

\section{Error analysis}

\chapter{Summary}
We have demonstrated a nonprojective graph-based dependency parser implemented as a single deep
neural network that directly produces parse trees from characters and does not require
other NLP tools such as a POS tagger. The parser is trained in an end-to-end manner,
and has separate cost terms that pertain to label accuracy,
head word localization and optionally POS tagging. On
morphologically rich languages the parser is competitive
with other state-of-the-art parsers.

With additional experiments we have established that the multilingual training
performance depends heavily on degree of parameter sharing, and can differ
depending on language similarity and corpus size. The decoding
algorithm used does not have a big impact on the performance of the system,
so simple greedy one is sufficient.

%%%%% BIBLIOGRAFIA

%\bibliographystyle{plain}
\bibliographystyle{apalike}
\bibliography{thesis}

%\begin{thebibliography}{1}
%\bibitem{example} \ldots
%\end{thebibliography}

\end{document}
